\documentclass{lmcs} %%% last changed 2014-08-20

%% mandatory lists of keywords 
%\keywords{dependent type theory, presheaf models, modal type theory, homotopy type theory, parametricity, directed type theory, guarded type theory}

%% read in additional TeX-packages or personal macros here:
%% e.g. \usepackage{tikz}
%\usepackage{libertine}
%\usepackage[libertine]{newtxmath}

%\usepackage[T1]{fontenc}
%\usepackage{lmodern}
	% This is to avoid the error:
	% "! pdfTeX error (font expansion): auto expansion is only possible with scalable fonts."
                     
%\usepackage{amsthm}
\usepackage{hyperref}
\usepackage{cleveref}
\usepackage{xypic}
%%\input{myMacros.tex}
%% define non-standard environments BEYOND the ones already supplied 
%% here, for example
\theoremstyle{plain}\newtheorem{satz}[thm]{Satz} %\crefname{satz}{Satz}{S\"atze}
%% Do NOT replace the proclamation environments lready provided by
%% your own.

\allowdisplaybreaks

% --------------------------------------------------
% Get cleveref to work: https://tex.stackexchange.com/a/354706/43137

\newcommand{\renewtheorem}[1]{%
  \expandafter\let\csname #1\endcsname\relax
  \expandafter\let\csname c@#1\endcsname\relax
  \expandafter\let\csname end#1\endcsname\relax
  \newtheorem{#1}%
}

\theoremstyle{plain}

\renewtheorem{thm}{Theorem}[section]
\renewtheorem{cor}[thm]{Corollary}
\renewtheorem{lem}[thm]{Lemma}
\renewtheorem{prop}[thm]{Proposition}
\renewtheorem{asm}[thm]{Assumption}

% the following environments keep the roman font:
\theoremstyle{definition}

\renewtheorem{rem}[thm]{Remark}
\renewtheorem{rems}[thm]{Remarks}
\renewtheorem{exa}[thm]{Example}
\renewtheorem{exas}[thm]{Examples}
\renewtheorem{nota}[thm]{Notation}
\renewtheorem{defi}[thm]{Definition}
\renewtheorem{conv}[thm]{Convention}
\renewtheorem{conj}[thm]{Conjecture}
\renewtheorem{prob}[thm]{Problem}
\renewtheorem{oprob}[thm]{Open Problem}
\renewtheorem{algo}[thm]{Algorithm}
\renewtheorem{obs}[thm]{Observation}
\renewtheorem{qu}[thm]{Question}
\renewtheorem{fact}[thm]{Fact}
\renewtheorem{pty}[thm]{Property}

% --------------------------------------------------

%\newenvironment{proposition}[1][]{\begin{prop}[#1]}{\end{prop}}
%\newenvironment{theorem}[1][]{\begin{thm}[#1]}{\end{thm}}
%\newenvironment{definition}[1][]{\begin{defi}[#1]}{\end{defi}}
%\newenvironment{example}[1][]{\begin{exa}[#1]}{\end{exa}}
%\newenvironment{remark}[1][]{\begin{rem}[#1]}{\end{rem}}
%\newenvironment{notation}[1][]{\begin{nota}[#1]}{\end{nota}}
%\newenvironment{corollary}[1][]{\begin{cor}[#1]}{\end{cor}}
\newenvironment{proposition}[0]{\begin{prop}}{\end{prop}}
\newenvironment{theorem}[0]{\begin{thm}}{\end{thm}}
\newenvironment{definition}[0]{\begin{defi}}{\end{defi}}
\newenvironment{example}[0]{\begin{exa}}{\end{exa}}
\newenvironment{remark}[0]{\begin{rem}}{\end{rem}}
\newenvironment{notation}[0]{\begin{nota}}{\end{nota}}
\newenvironment{corollary}[0]{\begin{cor}}{\end{cor}}

%\def\eg{{\em e.g.}}
%\def\cf{{\em cf.}}

% ---------------------------------------------------------

%\newcommand{\todoi}[1]{\textbf{\textcolor{red}{#1 \qed}}\par\noindent}
%\newcommand{\todo}[1]{\textbf{\textcolor{red}{\footnote{\textcolor{red}{#1}}}}}
%\newcommand{\domii}[1]{\textbf{\textcolor{green!70!black}{Domi: #1 \qed}}\par\noindent}
%\newcommand{\domi}[1]{\textbf{\textcolor{green!70!black}{\footnote{\textcolor{green!70!black}{Domi: #1}}}}}
%\newcommand{\revi}[2]{\textbf{\textcolor{blue}{Rev #1: #2 \qed}}\par\noindent}
%\newcommand{\rev}[2]{\textbf{\textcolor{blue}{\footnote{\textcolor{blue}{Rev #1: #2}}}}}
%\newcommand{\okrev}[2]{\textbf{\textcolor{purple}{\footnote{\textcolor{purple}{OK Rev #1: #2}}}}}
%\newcommand{\oktodo}[1]{\textbf{\textcolor{purple}{\footnote{\textcolor{purple}{#1}}}}}
%\newcommand{\todoi}[1]{}
%\newcommand{\todo}[1]{}
%\newcommand{\domii}[1]{}
%\newcommand{\domi}[1]{}
%\newcommand{\revi}[2]{}
%\newcommand{\rev}[2]{}
%\newcommand{\okrev}[2]{}
%\newcommand{\oktodo}[1]{}

% ---------------------------------------------------------

\usepackage[todos]{../natpt-macros}

% ---------------------------------------------------------

%% due to the dependence on amsart.cls, \begin{document} has to occur
%% BEFORE the title and author information:

\begin{document}

\title[Higher Pro-Arrows: From Semantics towards Syntax of Naturality Pretype Theory]{Higher Pro-Arrows: \\ From Semantics towards Syntax \\ of Naturality Pretype Theory}
%\title[Instructions]{Instructions for Authors\\How to prepare papers
%  for LMCS using \texorpdfstring{\MakeLowercase{\texttt{lmcs.cls}}}{lmcs.cls}\rsuper*\\Version of 
%  2014-02-01}
%\titlecomment{{\lsuper*}OPTIONAL comment concerning the title, \eg, 
%  if a variant or an extended abstract of the paper has appeared elsewhere.}

\author[Andreas Nuyts]{Andreas Nuyts\lmcsorcid{0000-0002-1571-5063}}	%required
\thanks{Andreas Nuyts holds a Postdoctoral Fellowship from the Research Foundation - Flanders (FWO; 1247922N, 12AB225N). This research is partially funded by the Research Fund KU Leuven.}	%optional

\address{DistriNet, KU Leuven, Belgium}	%required
\email{andreas.nuyts@kuleuven.be}  %optional
\urladdr{anuyts.github.io}  %optional
%\email{name1@email1}  %optional

%\author[C.~Name3]{Carla Name3}	%optional
%\address{address 3}	%optional
%\urladdr{name3@url3\quad\rm{(optionally, a web-page can be specified)}}  %optional
%\thanks{thanks 3, optional.}	%optional

%% etc.

%% required for running head on odd and even pages, use suitable
%% abbreviations in case of long titles and many authors:

%%%%%%%%%%%%%%%%%%%%%%%%%%%%%%%%%%%%%%%%%%%%%%%%%%%%%%%%%%%%%%%%%%%%%%%%%%%

%% the abstract has to PRECEDE the command \maketitle:
%% be sure not to issue the \maketitle command twice!

\begin{abstract}
\noindent 
In systems with internal parametricity,
relations propagate through type-level functions and are preserved by term-level functions for free.
In homotopy type theory (HoTT), we get preservation of equivalences by all functions for free.
In directed type theory, we get preservation of morphisms by all (covariant) functions for free.
None of these three properties by itself is satisfactory:
HoTT and directed type theory are less general in the sense that equivalences and morphisms give rise to a (graph) relation.
However, parametricity is less powerful in that relations do not carry the computational information of morphisms.
Finally, directed type theory requires functoriality of the operation we propagate through.

We set out to develop a type system that has all three preservation properties in an interactive manner, so that we can preserve isomorphisms when available, morphisms when functorial, and relations as a last resort.
We use modalities to keep track of variance of functions as we build them.
Such a system should provide us with functoriality (\texttt{fmap}), parametricity and naturality proofs for free. I call such a system Naturality Type Theory (NatTT).
In this first step, I consider Naturality \emph{Pre}type Theory (NatPT):
I defer all considerations of fibrancy to intuition and future work,
including in particular the specifics of composition of and transport along morphisms.

By instantiating parametrized systems such as Multimod(e/al) Type Theory (MTT) and the Modal Transpension System (MTraS), we can moreover separate concerns and only worry about the presheaf model at every mode, and the modalities that we can model as adjunctions between these presheaf models, leaving syntactic matters to research on MTT and MTraS.
The presheaf models are designed to accommodate higher pro-arrow equipments (whose operations are to be specified in future work as part of a fibrancy condition), and can be invented in three ways: (1) as a higher-dimensional version of pro-arrow equipments, (2) as a heterogenization of Tamsamani and Simpson's model of higher category theory, and (3) as a directification of Degrees of Relatedness.

I introduce the 2-poset of polarized reshuffles, a variant of which will serve as the mode theory of NatPT, and build two models of MTT instantiated on this mode theory in presheaves called jet-topic and jet-cubical sets.
These presheaf categories are equipped with twisted prism endofunctors on which we can instantiate MTraS to obtain an internal twisted interval in NatPT.

Lacking a formal treatment of fibrancy as of yet, I use informal and necessarily somewhat speculative discourse to evaluate the viability of NatPT and its models as a basis for NatTT.
\end{abstract}

\maketitle

\section{Introduction}
a

\section*{Acknowledgements}
Thank you!!

\appendix

\bibliographystyle{alphaurl}
\bibliography{../refs/refs.bib}

%\appendix
%\section{}
%  Here is a check-list to be completed before submitting the paper to
%  LMCS:
%\begin{itemize}[label=$\triangleright$]
%\item your submission uses the latest version of lmcs.cls
%\item the text of your submission is contained in a single file,
%  except for macros and graphics
%\item your graphics use only one format 
%\item you have employed the Journal's original proclamation environments,
%  or suitable extensions thereof 
%\item you have loaded the hyperref package
%\item you have \emph{not} loaded the times package
%\item you have not routinely adjusted vertical spacing manually by issuing
%  \texttt{\textbackslash vspace} or \texttt{\textbackslash vskip} commands
%\item you have used the command \texttt{\textbackslash sloppy} only
%  locally and in emergency cases
%\item your displayed equations use the
%  \texttt{\textbackslash[\dots\textbackslash]} construct
%\item your abstract only contains as few math-expressions as possible and no
%  references 
%\end{itemize}
%
%  This listing also shows how to override the default bullet $\bullet$
%  of the \texttt{itemize}-envronment by a different symbol, in this
%  case \texttt{\textbackslash triangleright}.
\end{document}
