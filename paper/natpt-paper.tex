\documentclass{lmcs} %%% last changed 2014-08-20

%% mandatory lists of keywords 
%\keywords{dependent type theory, presheaf models, modal type theory, homotopy type theory, parametricity, directed type theory, guarded type theory}

%% read in additional TeX-packages or personal macros here:
%% e.g. \usepackage{tikz}
%\usepackage{libertine}
%\usepackage[libertine]{newtxmath}

%\usepackage[T1]{fontenc}
%\usepackage{lmodern}
	% This is to avoid the error:
	% "! pdfTeX error (font expansion): auto expansion is only possible with scalable fonts."
                     
%\usepackage{amsthm}
\usepackage{hyperref}
\usepackage{cleveref}
\usepackage{xypic}
%%\input{myMacros.tex}
%% define non-standard environments BEYOND the ones already supplied 
%% here, for example
\theoremstyle{plain}\newtheorem{satz}[thm]{Satz} %\crefname{satz}{Satz}{S\"atze}
%% Do NOT replace the proclamation environments lready provided by
%% your own.

\allowdisplaybreaks

% --------------------------------------------------
% Get cleveref to work: https://tex.stackexchange.com/a/354706/43137

\newcommand{\renewtheorem}[1]{%
  \expandafter\let\csname #1\endcsname\relax
  \expandafter\let\csname c@#1\endcsname\relax
  \expandafter\let\csname end#1\endcsname\relax
  \newtheorem{#1}%
}

\theoremstyle{plain}

\renewtheorem{thm}{Theorem}[section]
\renewtheorem{cor}[thm]{Corollary}
\renewtheorem{lem}[thm]{Lemma}
\renewtheorem{prop}[thm]{Proposition}
\renewtheorem{asm}[thm]{Assumption}

% the following environments keep the roman font:
\theoremstyle{definition}

\renewtheorem{rem}[thm]{Remark}
\renewtheorem{rems}[thm]{Remarks}
\renewtheorem{exa}[thm]{Example}
\renewtheorem{exas}[thm]{Examples}
\renewtheorem{nota}[thm]{Notation}
\renewtheorem{defi}[thm]{Definition}
\renewtheorem{conv}[thm]{Convention}
\renewtheorem{conj}[thm]{Conjecture}
\renewtheorem{prob}[thm]{Problem}
\renewtheorem{oprob}[thm]{Open Problem}
\renewtheorem{algo}[thm]{Algorithm}
\renewtheorem{obs}[thm]{Observation}
\renewtheorem{qu}[thm]{Question}
\renewtheorem{fact}[thm]{Fact}
\renewtheorem{pty}[thm]{Property}

% --------------------------------------------------

%\newenvironment{proposition}[1][]{\begin{prop}[#1]}{\end{prop}}
%\newenvironment{theorem}[1][]{\begin{thm}[#1]}{\end{thm}}
%\newenvironment{definition}[1][]{\begin{defi}[#1]}{\end{defi}}
%\newenvironment{example}[1][]{\begin{exa}[#1]}{\end{exa}}
%\newenvironment{remark}[1][]{\begin{rem}[#1]}{\end{rem}}
%\newenvironment{notation}[1][]{\begin{nota}[#1]}{\end{nota}}
%\newenvironment{corollary}[1][]{\begin{cor}[#1]}{\end{cor}}
\newenvironment{proposition}[0]{\begin{prop}}{\end{prop}}
\newenvironment{theorem}[0]{\begin{thm}}{\end{thm}}
\newenvironment{definition}[0]{\begin{defi}}{\end{defi}}
\newenvironment{example}[0]{\begin{exa}}{\end{exa}}
\newenvironment{remark}[0]{\begin{rem}}{\end{rem}}
\newenvironment{notation}[0]{\begin{nota}}{\end{nota}}
\newenvironment{corollary}[0]{\begin{cor}}{\end{cor}}

%\def\eg{{\em e.g.}}
%\def\cf{{\em cf.}}

% ---------------------------------------------------------

%\newcommand{\todoi}[1]{\textbf{\textcolor{red}{#1 \qed}}\par\noindent}
%\newcommand{\todo}[1]{\textbf{\textcolor{red}{\footnote{\textcolor{red}{#1}}}}}
%\newcommand{\domii}[1]{\textbf{\textcolor{green!70!black}{Domi: #1 \qed}}\par\noindent}
%\newcommand{\domi}[1]{\textbf{\textcolor{green!70!black}{\footnote{\textcolor{green!70!black}{Domi: #1}}}}}
%\newcommand{\revi}[2]{\textbf{\textcolor{blue}{Rev #1: #2 \qed}}\par\noindent}
%\newcommand{\rev}[2]{\textbf{\textcolor{blue}{\footnote{\textcolor{blue}{Rev #1: #2}}}}}
%\newcommand{\okrev}[2]{\textbf{\textcolor{purple}{\footnote{\textcolor{purple}{OK Rev #1: #2}}}}}
%\newcommand{\oktodo}[1]{\textbf{\textcolor{purple}{\footnote{\textcolor{purple}{#1}}}}}
%\newcommand{\todoi}[1]{}
%\newcommand{\todo}[1]{}
%\newcommand{\domii}[1]{}
%\newcommand{\domi}[1]{}
%\newcommand{\revi}[2]{}
%\newcommand{\rev}[2]{}
%\newcommand{\okrev}[2]{}
%\newcommand{\oktodo}[1]{}

% ---------------------------------------------------------

\usepackage[todos]{../natpt-macros}

% ---------------------------------------------------------

%% due to the dependence on amsart.cls, \begin{document} has to occur
%% BEFORE the title and author information:

\begin{document}

\title[Higher Pro-Arrows: From Semantics towards Syntax of Naturality Pretype Theory]{Higher Pro-Arrows: \\ From Semantics towards Syntax \\ of Naturality Pretype Theory}
%\title[Instructions]{Instructions for Authors\\How to prepare papers
%  for LMCS using \texorpdfstring{\MakeLowercase{\texttt{lmcs.cls}}}{lmcs.cls}\rsuper*\\Version of 
%  2014-02-01}
%\titlecomment{{\lsuper*}OPTIONAL comment concerning the title, \eg, 
%  if a variant or an extended abstract of the paper has appeared elsewhere.}

\author[Andreas Nuyts]{Andreas Nuyts\lmcsorcid{0000-0002-1571-5063}}	%required
\thanks{Andreas Nuyts holds a Postdoctoral Fellowship from the Research Foundation - Flanders (FWO; 1247922N, 12AB225N). This research is partially funded by the Research Fund KU Leuven.}	%optional

\address{DistriNet, KU Leuven, Belgium}	%required
\email{andreas.nuyts@kuleuven.be}  %optional
\urladdr{anuyts.github.io}  %optional
%\email{name1@email1}  %optional

%\author[C.~Name3]{Carla Name3}	%optional
%\address{address 3}	%optional
%\urladdr{name3@url3\quad\rm{(optionally, a web-page can be specified)}}  %optional
%\thanks{thanks 3, optional.}	%optional

%% etc.

%% required for running head on odd and even pages, use suitable
%% abbreviations in case of long titles and many authors:

%%%%%%%%%%%%%%%%%%%%%%%%%%%%%%%%%%%%%%%%%%%%%%%%%%%%%%%%%%%%%%%%%%%%%%%%%%%

%% the abstract has to PRECEDE the command \maketitle:
%% be sure not to issue the \maketitle command twice!

\begin{abstract}
\noindent 
In parametric, directed and homotopy type theory, we get preservation of relations, morphisms and equivalences (resp.) for free.
None of these systems is superior to the other: while relations are strictly more general than morphisms and equivalences, we lose computational content when we only consider relations.
Meanwhile, preservation of morphisms is subject to the rather strong condition of functoriality.

We set out to develop a type system where these three properties interact: functions have a modality/variance annotation that expresses how they act on relations, morphisms and equivalences.
%that has all three preservation properties in an interactive manner, so that we can preserve isomorphisms when available, morphisms when functorial, and relations as a last resort.
%We use modalities to keep track of variance of functions.
Such a system should provide us with functoriality, parametricity and naturality proofs for free.
I call such a system Naturality Type Theory (NatTT).

In this first step, I consider Naturality \emph{Pre}type Theory (NatPT):
I defer all formal considerations of fibrancy to future work,
including in particular the specifics of composition of and transport along morphisms.
%
By instantiating parametrized systems such as Multimod(e/al) Type Theory (MTT) and the Modal Transpension System (MTraS), we can moreover separate concerns and only worry about the semantics, leaving syntactic matters to research on MTT and MTraS.
I construct an MTT mode theory with a presheaf model designed to accommodate higher pro-arrow equipments (to be fully defined in future work), which can be invented in three ways: (1) as a higher-dimensional version of pro-arrow equipments, (2) as a heterogenization of Tamsamani and Simpson's model of higher category theory, and (3) as a directification of Degrees of Relatedness.
A twisted prism functor on the base categories models an internal twisted interval in the sense of MTraS.
Finally, I discuss how the matter of fibrancy can be addressed in the future.
\end{abstract}

\maketitle

%\todoi{``In general, the length of the abstract should not exceed 20 lines''}

\section{Old Abstract}
\todoi{Remove this section after it's been fully subsumed in the intro}
In systems with internal parametricity,
relations propagate through type-level functions and are preserved by term-level functions for free.
In homotopy type theory (HoTT), we get preservation of equivalences by all functions for free.
In directed type theory, we get preservation of morphisms by all (covariant) functions for free.
None of these three properties by itself is satisfactory:
HoTT and directed type theory are less general in the sense that equivalences and morphisms give rise to a (graph) relation, but not vice versa.
However, parametricity is less powerful in that relations do not carry the computational information of morphisms.
Finally, directed type theory requires functoriality of the operation we propagate through.

We set out to develop a type system that has all three preservation properties in an interactive manner, so that we can preserve isomorphisms when available, morphisms when functorial, and relations as a last resort.
We use modalities to keep track of variance of functions as we build them.
Such a system should provide us with functoriality (\texttt{fmap}), parametricity and naturality proofs for free. I call such a system Naturality Type Theory (NatTT).
In this first step, I consider Naturality \emph{Pre}type Theory (NatPT):
I defer all formal considerations of fibrancy to future work,
including in particular the specifics of composition of and transport along morphisms.

By instantiating parametrized systems such as Multimod(e/al) Type Theory (MTT) and the Modal Transpension System (MTraS), we can moreover separate concerns and only worry about the presheaf model at every mode, and the modalities that we can model as adjunctions between these presheaf models, leaving syntactic matters to research on MTT and MTraS.
The presheaf models are designed to accommodate higher pro-arrow equipments (some of whose operations are to be specified in future work as part of a fibrancy condition), and can be invented in three ways: (1) as a higher-dimensional version of pro-arrow equipments, (2) as a heterogenization of Tamsamani and Simpson's model of higher category theory, and (3) as a directification of Degrees of Relatedness.

I introduce the 2-poset of polarized reshuffles, a variant of which will serve as the mode theory of NatPT, and build two models of MTT instantiated on this mode theory in presheaves called jet-topic and jet-cubical sets.
The base categories are equipped with twisted prism endofunctors on which we can instantiate MTraS to obtain an internal twisted interval in NatPT.

Lacking a formal treatment of fibrancy as of yet, I use informal and necessarily somewhat speculative discourse to evaluate the viability of NatPT and its models as a basis for NatTT.

\section{Introduction}

\subsection{Category Theory in Functional Programming}
Several decades of functional programming practice in simply, polymorphically and dependently typed programming languages has made it abundantly clear that there is virtue in identifying algebraic and categorical structures in program designs, as these abstractions enable code reuse by relying on abstract libraries for algebra and category theory.\todo{Cite something?}

While the design of Haskell, its libraries and tutorials can create the impression that $\Type$, the category of types and functions, is the \emph{only} category we should care about, it is by now clear that this is absolutely not the case.
To only scratch the surface of this fact:
\begin{itemize}
	\item Pure functional programming languages tend to use monads\todo{Footnote about algebraic effects?} as a mathematical device for allowing non-purity \cite{moggi-monads}, where specific monads only allow for specific forms of side-effects.
	\textbf{Monad morphisms} are then effect reinterpretations, explaining the effects allowed by one monad in terms of the effects allowed by another.\todo{Cite?}
	When monads are being combined using monad transformers\todo{Cite!}, then monad morphisms may arise from \textbf{morphisms of monad transformers}.
	When monads (or transformers) are indexed by other structures (e.g.\ the writer monad $\Writer\,W = W \times \loch$, which allows programs to log messages, is indexed by a monoid of messages $W$), then monad morphisms arise from \textbf{morphisms between such structures}.
	
	\item Inductive data types are now understood as initial algebras $\mu F$ of some polynomial functor (a.k.a.\ container functor) $F$ \cite{containers}, i.e.\ $\mu F$ is the initial object in the \textbf{category of $F$-algebras}.
	The data needed to define a function by recursion on inductive data exactly constitutes an $F$-algebra $A$, with the motive being the carrier and each constructor clause being an algebraic operation.
	The function $f : \mu F \to A$ thus defined is the unique algebra morphism from the initial object $\mu F$, while the $\beta$-rules for $\mu F$ exactly ensure that it is indeed an algebra morphism.
	Dually, codata types are final coalgebras.
	
	\item Optics such as lenses and traversals are bidirectional data accessors which can, e.g., read and update fields of record types (lenses) or entries of traversable functors (traversals).\todo{Cite nLab?}
	Optics themselves have identities and can be composed, \textbf{thus forming categories}, but they also have slick parametric encodings called Van Laarhoven optics\todo{Cite VL? Cite Haskell library?} and profunctor optics\todo{Cite some paper? Cite Haskell library?}, correctness of which relies on parametric/natural quantification over \textbf{categories of (pro)functors with extra structure}.
\end{itemize}

\subsection{Propagation and Preservation for Free}
Beyond library support for category theory, three strains of research seek to provide \emph{language} support for (adaptations of) category theory.

\subsubsection{Directed Type Theory}
In Directed Type Theory (DirTT) \cite{2dtt,riehl-shulman-dtt,north-dirhott,dua-simplicial}\todo{Check Gratzer's paper.}, all types $A$ are regarded as categories and typically come equipped natively with an indexed type $\Hom_A(a, b)$ of morphisms from $a : A$ to $b : A$.
Very na\"ively, we could hope for these morphisms to be preserved by \emph{all} functions, meaning that all functions would be covariant functors.
However, as anyone who has ever touched category knows, there are many interesting operation that are not a functor at all.

\subsubsection{Parametricity}

\todoi{Old Stuff}

We consider three strains of research that all seek to unburden users of proof assistants: parametricity, homotopy type theory (HoTT), and directed type theory (DirTT).


\todoi{Parametricity}
\todoi{HoTT}
\todoi{Directed TT}
\todoi{Everything has a graph: HoTT $<$ DirTT $<$ parametricity}
\todoi{Computational content: parametricity $<$ HoTT and DirTT}
\todoi{Functoriality: DirTT $<$ HoTT and parametricity}

\todoi{You don't have to give a literature overview of parametricity here}

\subsection{Motivation}

\subsection{Example Problem in Functional Programming}

\subsection{Overview of the Paper}

\section{A Quick Look at Fibrancy (Before Setting It Aside)}
\todoi{Related work on internalizing fibrancy}

\section{Separation of Concerns: A Model-First Approach}

\section{Prerequisites}
\subsection{Models of Type Theory}
\todoi{Discuss GATs}
\todoi{Quick intro of CwFs}
\todoi{Mention natural models}

\subsection{Presheaves and Presheaf Models of Type Theory}
\todoi{Superquick intro to presheaves and their relevance to preservation properties}
\todoi{Justify cubical/simplicial stuff as higher graphs}
\todoi{Quickly review presheaf model of TT}

\subsection{Multimod(e/al) Type Theory (MTT)}
\todoi{Either from transpension paper or based on your slides, or both}

\subsection{Lifting Functors to Presheaf Categories}

\subsection{The Modal Transpension System (MTraS)}

\section{Three Approaches to the Model}

\subsection{Tamsamani \& Simpson's Model of $n$-Categories}

\subsection{Pro-Arrow Equipments}

\subsection{Higher Pro-Arrow Equipments as Heterogenized $n$-Categories}

\subsection{Degrees of Relatedness}

\subsection{Directifying Degrees of Relatedness}

\section{Modes and Base Categories}

\subsection{Modes are Anpolarity Masks}
\todoi{Storeys and degrees}

\subsection{Jet Sets and Jet Setoids}

\subsection{Jet Topes and Jet-Topic Sets}

\subsection{Jet Cubes and Jet-Cubical Sets}

\section{Modalities are Polarized Reshuffles}

\subsection{The 2-Poset of Polarized Reshuffles}

\subsection{Semantics on Jet Sets and Jet Setoids}

\subsection{Semantics on Jet Topes and Jet Cubes}

\subsection{Lifting to Presheaves: Semantics on Jet-Topic and Jet-Cubical Sets}

\section{Pruning the Modes: Paths, Jets and Bridges}

\section{Modalities in Action}

\subsection{The Natural Universe}

\subsection{Functoriality}

\subsection{Limithood and Naturality}

\subsection{Modal Existentials}

\section{Intervals and the Hom-Type}
\todoi{Symmetric and Twisted intervals}
\todoi{Properties of the multipliers and internalization}
\todoi{Extension types?}
\todoi{Arguments for bump: (1) reason about pro-arrows, (2) identity should be natural, not mixed, (3) univalence + bump in universe}
\todoi{Hom-type}
\todoi{Path type (also requires a bump)}
\todoi{Bridge type (does not require a bump)}
\todoi{Relational univalence (a.k.a.\ relativity)}

\section{Evaluation and Outlook: Fibrancy}
\todoi{Admit non-formalness and speculativity again}

\subsection{Functorial Types}
\todoi{directed univalence}
\todoi{DUA UP TO A SECTION IS ENOUGH BECAUSE OF THE BUMP}

\subsection{Kan Types}
\todoi{What do we know about symmetric jet-cubes?}
\todoi{Does shape substitution commute with transpension? Yes if multiplier is a top-slice right adjoint, see transpension techreport.}
\todoi{Can we just say: the heterogeneous path type is covariant?}
\todoi{Can we just say: the path-localization is Segal?}

\subsection{Segal Types}
\todoi{Can we just say: the heterogeneous Hom-type is covariant?}
\todoi{Open boxes where the walls are determined by a covariant proposition on the floor / contravariant proposition on the ceiling?}

\subsection{Rezk Types}
\todoi{categorical univalence}
\todoi{$\name{cua} \circ \name{dua} = \name{ua}$}
\todoi{Other Rezk-like conditions for different storey-composition}

\subsection{Conduch\'e Types}
\todoi{At least briefly mention the considerworthiness of it.}

\subsection{Contextual Fibrancy}
\todoi{Via information flow modalities, perhaps mention embargoes.}

\section{Known Hazards in Directed Type Theory}
\todoi{
\begin{itemize}
	\item The Conduch\'e issue (but what's the issue again?)
	\item $\catC\op \times \catC$
	\item Mixed-variant $\id$
	\item Coupling of type/term variance
\end{itemize}
}
\todoi{Also discuss somewhere: should path $\to$ jet $\to$ bridge be in the base category? The alternative is to have Kan and Segal open boxes with different dimension flavours on the floor, but this will become extremely complex.}
\todoi{Kock: http://home.math.au.dk/kock/fib9.pdf}

\section{Related Work}
\todoi{Lots of stuff}
\todoi{Paige}
\todoi{Riehl/Shulman}
\todoi{Weaver/Licata}
\todoi{Greta Coraglia}

\section*{Acknowledgements}
\todoi{Acknowledgements}

\appendix

\bibliographystyle{alphaurl}
\bibliography{../refs/refs.bib}

%\appendix
%\section{}
%  Here is a check-list to be completed before submitting the paper to
%  LMCS:
%\begin{itemize}[label=$\triangleright$]
%\item your submission uses the latest version of lmcs.cls
%\item the text of your submission is contained in a single file,
%  except for macros and graphics
%\item your graphics use only one format 
%\item you have employed the Journal's original proclamation environments,
%  or suitable extensions thereof 
%\item you have loaded the hyperref package
%\item you have \emph{not} loaded the times package
%\item you have not routinely adjusted vertical spacing manually by issuing
%  \texttt{\textbackslash vspace} or \texttt{\textbackslash vskip} commands
%\item you have used the command \texttt{\textbackslash sloppy} only
%  locally and in emergency cases
%\item your displayed equations use the
%  \texttt{\textbackslash[\dots\textbackslash]} construct
%\item your abstract only contains as few math-expressions as possible and no
%  references 
%\end{itemize}
%
%  This listing also shows how to override the default bullet $\bullet$
%  of the \texttt{itemize}-envronment by a different symbol, in this
%  case \texttt{\textbackslash triangleright}.
\end{document}
