\documentclass[a4paper]{article}

\usepackage{amsthm}
\pdfoutput=1
\usepackage{libertine}
\usepackage[libertine]{newtxmath}
\usepackage{a4wide}

%\usepackage{mathpartir}
%\usepackage{natbib}
%\usepackage{tipa}

%\usepackage{geometry}
%\newcommand{\todo}[1]{\textbf{\red{*}}\marginpar{\red{#1}}}
%\renewcommand{\todoi}[1]{\par \textbf{\red{#1}} \par}

%\usepackage{amsmath}
%\usepackage{amssymb}
%\usepackage{thmtools}
%\usepackage{mathrsfs}
%%\usepackage{mathabx}
%\usepackage{latexsym}
\usepackage[hidelinks,bookmarksnumbered]{hyperref}
\usepackage{cleveref}
\usepackage{xcolor}
\usepackage{xypic}
	\CompileMatrices
%\usepackage{stmaryrd}
%\usepackage{epstopdf}
%\usepackage{array}
%\usepackage[normalem]{ulem}

\usepackage[english]{babel}
%\usepackage[dutch]{babel}

\usepackage{../natpt-macros}

%\newcommand{\todoi}[1]{\textbf{\textcolor{red}{#1 \qed}}\par\noindent}
%\newcommand{\todo}[1]{\textbf{\textcolor{red}{\footnote{\textcolor{red}{#1}}}}}
%\newcommand{\todoi}[1]{}
%\newcommand{\todo}[1]{}

\newcommand{\thetitle}{Naturality Pretype Theory - Extended Version}
\newcommand{\theauthors}{Andreas Nuyts}
\begin{document}
	\addtolength{\voffset}{-.5in}

\title{\thetitle}
%\subtitle{Technical report}
\date{\today}
\author{\theauthors{}}
%\address{imec-DistriNet, KU Leuven, Belgium}
\maketitle
%\vspace*{-.5in}
%\begin{abstract}
%\end{abstract}

\tableofcontents

\pagebreak

\section{Base Categories and Modes}

\subsection{Modes are Anpolarity Lists}
In RelDTT, modes were natural numbers (minus one) expressing the number of available relations.
In NatPT, we will specify for each of these relations whether it is directed or not.
\begin{definition} \label{def:anpolarity}
	An anpolarity\footnote{`An' is Latin for `whether'.} is an element of the set $\Anpolarity := \accol{\ypolar, \npolar}$.
\end{definition}
\noindent As our set of modes, we take $\List\,\Anpolarity$.

\begin{remark}
	We note that it is always possible to restrict our mode theory, by discarding modes but keeping the same modalities and 2-cells between remaining modes.
	We could decide to restrict to any of the following subsets of modes:
	\begin{itemize}
		\item Modes of the form $[\ypolar]^*$, i.e.\ where all degrees are polar,
		\item Modes of the form $([\npolar] | [\npolar, \ypolar])^*$, i.e.\ where we think of a level as containing a path relation and optionally a weaker jet relation,
		\begin{itemize}
			\item Modes of the form $[\npolar, \ypolar]^*$ where the presence of a jet relation at each level is required,
		\end{itemize}
		\item Modes of the form $([\npolar] | [\ypolar, \npolar])^*$, i.e.\ where we think of a level as containing a bridge relation and optionally a stronger jet relation,
		\begin{itemize}
			\item Modes of the form $[\ypolar, \npolar]^*$ where the presence of a jet relation at each level is required,
		\end{itemize}
		\item Modes of the form $([\npolar, \npolar] | [\npolar, \ypolar, \npolar])^*$, i.e.\ where we think of a level as containing a path relation and a weaker bridge relation and optionally, in between, a jet relation,
		\begin{itemize}
			\item Modes of the form $[\npolar, \ypolar, \npolar]^*$ where the presence of a jet relation at each level is required.
		\end{itemize}
	\end{itemize}
	We will occasionally discuss these subtheories. By considering all of $\List\,\Anpolarity$ in the current paper, we maintain generality.
\end{remark}

\bibliographystyle{alphaurl}
\bibliography{../relyoneda-refs.bib}

\end{document}