\documentclass[a4paper]{article}

\usepackage{a4wide}

\usepackage{amsthm}
\pdfoutput=1

\RequirePackage{amsmath}
\RequirePackage{amssymb}
\RequirePackage{mathrsfs}
\RequirePackage{mathtools}
%\RequirePackage{mathabx}
\RequirePackage{latexsym}
%\RequirePackage{epstopdf}
%\RequirePackage{array}
\RequirePackage{stmaryrd}
\RequirePackage[normalem]{ulem}
%\RequirePackage{xfrac}

%\usepackage{bm}
\usepackage[all,cmtip]{xy}
%\usepackage{mathtools}
%\usepackage{longtable}

%\usepackage{mdframed}
%\let\framedCopy=\framed
%\let\framed=\undefined

\usepackage[hidelinks,bookmarksnumbered]{hyperref}
\RequirePackage{cleveref}

\usepackage{wasysym}
\usepackage{stmaryrd}
%\usepackage[notext]{stix}
\usepackage{fontawesome}
\RequirePackage{amssymb}
\RequirePackage{xcolor}
\usepackage{xypic}
	\CompileMatrices
\RequirePackage{mathtools}
\usepackage{stmaryrd}

\usepackage{libertine}

\usepackage[english]{babel}
%\usepackage[dutch]{babel}

\usepackage{../natpt-macros}

%\newcommand{\todoi}[1]{\textbf{\textcolor{red}{#1 \qed}}\par\noindent}
%\newcommand{\todo}[1]{\textbf{\textcolor{red}{\footnote{\textcolor{red}{#1}}}}}
%\newcommand{\todoi}[1]{}
%\newcommand{\todo}[1]{}

\newcommand{\thetitle}{Naturality Pretype Theory - Extended Version}
\newcommand{\theauthors}{Andreas Nuyts}
\begin{document}
	\addtolength{\voffset}{-.5in}

\title{\thetitle}
%\subtitle{Technical report}
\date{\today}
\author{\theauthors{}}
%\address{imec-DistriNet, KU Leuven, Belgium}
\maketitle
%\vspace*{-.5in}
%\begin{abstract}
%\end{abstract}

\tableofcontents

\pagebreak

\section{Base Categories and Modes}

\subsection{Modes are Anpolarity Masks}
In RelDTT, modes were natural numbers (minus one) expressing the number of available relations.
In NatPT, we will specify for each of these relations whether it is directed or not.
\begin{definition} \label{def:anpolarity}
	An \textbf{anpolarity}\footnote{`An' is Latin for `whether', as in `Nescio an polare sit,' meaning `I do not know whether it is polar'.} is an element of the set $\Anpolarity := \accol{\ypolar, \npolar}$, where $\ypolar$ stands for polar/directed and $\npolar$ stands for nonpolar/symmetric.
	
	An \textbf{anpolarity mask} or just \textbf{mask} is a list $\vec a \in \List\,\Anpolarity$ of anpolarities. We write $\lenpol{\vec a}$ for its length.
\end{definition}

\begin{remark} \todoi{Move this remark}
	We note that it is always possible to restrict our mode theory, by discarding modes but keeping the same modalities and 2-cells between remaining modes.
	We could decide to restrict to any of the following subsets of modes:
	\begin{itemize}
		\item Modes of the form $[\ypolar]^*$, i.e.\ where all degrees are polar,
		\item Modes of the form $([\npolar] | [\npolar, \ypolar])^*$, i.e.\ where we think of a level as containing a path relation and optionally a weaker jet relation,
		\begin{itemize}
			\item Modes of the form $[\npolar, \ypolar]^*$ where the presence of a jet relation at each level is required,
		\end{itemize}
		\item Modes of the form $([\npolar] | [\ypolar, \npolar])^*$, i.e.\ where we think of a level as containing a bridge relation and optionally a stronger jet relation,
		\begin{itemize}
			\item Modes of the form $[\ypolar, \npolar]^*$ where the presence of a jet relation at each level is required,
		\end{itemize}
		\item Modes of the form $([\npolar, \npolar] | [\npolar, \ypolar, \npolar])^*$, i.e.\ where we think of a level as containing a path relation and a weaker bridge relation and optionally, in between, a jet relation,
		\begin{itemize}
			\item Modes of the form $[\npolar, \ypolar, \npolar]^*$ where the presence of a jet relation at each level is required.
		\end{itemize}
	\end{itemize}
	We will occasionally discuss these subtheories. By considering all of $\List\,\Anpolarity$ in the current paper, we maintain generality.
\end{remark}

\subsection{Jet Sets}
\begin{definition} \label{def:jetset}
	Let $\vec a$ be a mask. An \textbf{$\vec a$-jet set} is a set $X$ equipped with $\lenpol{\vec a}$ (proof-irrelevant%
	\footnote{So these relations are functions $X \to X \to \Prop$ where $\Prop$ is a universe of $h$-propositions \cite{hottbook}. In most applications, these relations will be decidable, but we do not require this.}%
	) relations $\jet_i$ where
	\begin{itemize}
		\item $0 \leq i < \lenpol{\vec a}$ is called the \textbf{degree},
		\item $\jet_i$ is called the \textbf{$i$-jet relation},
		\item its opposite $\tej_i$ is called the \textbf{opposite $i$-jet relation},
	\end{itemize}
	such that
	\begin{itemize}
		\item when $a_i = \npolar$, then $\jet_i$ is symmetric, in which case we will denote it as $\edge_i$ and call it the \textbf{$i$-edge relation} (notwithstanding that we still consider it a special case of a jet relation),
		\item $x \jet_i y$ implies both $x \jet_{i+1} y$ and $x \tej_{i+1} y$ whenever $0 \leq i < i+1 < \lenpol{\vec a}$.
	\end{itemize}
	A \textbf{morphism} of $\vec a$-jet sets is a function that preserves all the jet and edge relations.
	
	The category of $\vec a$-jet sets is called $\jetset{\vec a}$.
	
	A jet set is called \textbf{transitive} if each of the $i$-jet relations is transitive (i.e. a pre-order and, if $i = \npolar$, an equivalence relation).
\end{definition}
\begin{proposition}
	Let $X$ be a transitive $\vec a$-jet set and $0 \leq i < j < \lenpol{\vec a}$.
	Then the double category whose objects are elements of $X$, morphisms are (unique) proofs of $x \jet_i y$, pro-arrows are (unique) proofs of $x \jet_{j} y$ and squares are elements of the unit type, is a pro-arrow equipment \cite{nlab:proarrow,proarrow1,proarrow2}\todo{Refer to dedicated section instead.}.
\end{proposition}
\begin{proof}
	It is clearly a double category. The existence of companions and conjoints is trivial.
\end{proof}
\begin{definition}
	We define the
	\begin{itemize}
		\item \textbf{$i$-equijet relation} $\equijet_i$ as the symmetric interior of $\jet_i$, i.e.\ $x \equijet_i y$ if and only if $x \jet_i y$ and $x \tej_i y$;
		\item \textbf{$i$-infrajet relation} $\infrajet_i$ as the symmetric closure of $\jet_i$, i.e.\ $x \equijet_i y$ if and only if $x \jet_i y$ or $x \tej_i y$.
	\end{itemize}
\end{definition}
It is immediately clear that for nonpolar degrees, the jet/edge, equijet and infrajet relations coincide.
In general, we can observe that $x \infrajet_i y$ implies $x \equijet_j y$ for $i < j$.
So for mode $[\ypolar, \ypolar, \npolar]$, we get
\[
\xymatrix{
	& 
	\jet_0
		\ar@{:>}[rd]
	&&& 
	\jet_1 
		\ar@{:>}[rd]
	&&& 
	\jet_2 
		\ar@{<:>}[rd]
	\\
	\equijet_0
		\ar@{:>}[ru]
		\ar@{:>}[rd]
	&&
	\infrajet_0
		\ar@{:>}[r]
	&
	\equijet_1
		\ar@{:>}[ru]
		\ar@{:>}[rd]
	&&
	\infrajet_1
		\ar@{:>}[r]
	&
	\equijet_2
		\ar@{<:>}[ru]
		\ar@{<:>}[rd]
	&
	\edge_2
		\ar@{<:>}[l]
		\ar@{<:>}[u]
		\ar@{<:>}[r]
		\ar@{<:>}[d]
	&
	\infrajet_2.
	\\
	& 
	\tej_0 
		\ar@{:>}[ru]
	&&& 
	\tej_1 
		\ar@{:>}[ru]
	&&& 
	\tej_2 
		\ar@{<:>}[ru]
}
\]

\begin{definition} \label{def:op}
	Let $\vec a$ be a mask, $i < \lenpol{\vec a}$ and $X \in \Obj(\jetset{\vec a})$.
	We define the \textbf{$i$-opposite $\Op_i(X)$ of $X$} as the jet set with the same carrier and relations as $X$ except that the $i$-jet relation is reversed: $x \jet_i^{\Op_i(X)} y$ if and only if $x \tej_i^X y$.
	This defines a functor $\Op_i(X) : \jetset{\vec a} \to \jetset{\vec a}$.
\end{definition}
We have $\Op_i \circ \Op_i = \Id$ and if $a_i = \npolar$ then $\Op_i = \Id$.
\begin{definition} \label{def:sym-forget}
	Write $\vec a \sqsubset_i \vec b$ if $\lenpol{\vec a} = \lenpol{\vec b}$, $a_j = b_j$ for all $j \neq i$, $a_i = \npolar$ and $b_i = \ypolar$.
	
	If $\vec a \sqsubset_i \vec b$, then we write $\Usym_i : \jetset{\vec a} \to \jetset{\vec b}$ for the forgetful functor.
\end{definition}
\begin{corollary}
	The forgetful functor $\Usym_i$ as part of an adjoint triple $\Fsym_i \dashv \Usym_i \dashv \Csym_i$ where $\Fsym_i$ and $\Csym_i$ take a $\vec b$-jet set $X$ to a $\vec a$-jet set of the same carrier with the same $j$-jet relations for $j \neq i$ but where
	\begin{itemize}
		\item $x \edge_i^{\Fsym_i X} y$ if and only if $x \infrajet_i^X y$,
		\item $x \edge_i^{\Csym_i X} y$ if and only if $x \equijet_i^X y$.
	\end{itemize}
	We have $\Fsym_i \circ \Usym_i = \Csym_i \circ \Usym_i = \Id$, so that $\Usym_i \circ \Fsym_i$ is an idempotent monad and $\Usym_i \circ \Csym_i$ is an idempotent comonad.
\end{corollary}

\subsection{Intervals and Prisms}
\begin{definition} \label{def:interval}
	Let $\vec a$ be a mask and $i < \lenpol{\vec a}$.
	\begin{itemize}
		\item The \textbf{$i$-jet interval} $\rival{\jet_i}$ is defined as the jet set of mask $\vec a$ with carrier $\accol{0, 1}$ and relations generated by $0 \jet_i 1$.
		\item The \textbf{opposite $i$-jet interval} $\rival{\jet_i}$ is defined as the jet set of mask $\vec a$ with carrier $\accol{0, 1}$ and relations generated by $0 \tej_i 1$.
		\item The \textbf{$i$-equijet interval} $\rival{\equijet_i}$ is defined as the jet set of mask $\vec a$ with carrier $\accol{0, 1}$ and relations generated by $0 \equijet_i 1$.
	\end{itemize}
	If $a_i = \npolar$ then $\rival{\jet_i} = \rival{\tej_i} = \rival{\equijet_i} =: \rival{\edge_i}$ is called the \textbf{$i$-edge interval}.
\end{definition}
Note that it would be meaningless to define an $i$-infrajet interval in the same way.

\begin{definition} \label{def:prism}
	Let $\vec a$ be a mask, $i < \lenpol{\vec a}$ and $X \in \Obj(\jetset{\vec a})$.
	We define the \textbf{$i$-twisted prism $X \multip \rival{\jet_i}$} on $X$ as the jet set of mask $\vec a$ with
	\begin{itemize}
		\item carrier $X \times \accol{0, 1}$,
		\item jet relations generated by the following requirements:
		\begin{itemize}
			\item $(\loch, 0) : \Op_i(X) \to X \multip \rival{\jet_i}$ is a jet set morphism,
			\item $(\loch, 1) : X \to X \multip \rival{\jet_i}$ is a jet set morphism,
			\item $(x, 0) \jet_i (x, 1)$ for all $x \in X$.
		\end{itemize}
	\end{itemize}
	This defines the \textbf{$i$-twisted prism functor} $\loch \multip \rival{\jet_i} : \jetset{\vec a} \to \jetset{\vec a}$.
	
	We define the \textbf{opposite $i$-twisted prism $X \multip \rival{\tej_i}$} on $X$ as the jet set of mask $\vec a$ with
	\begin{itemize}
		\item carrier $X \times \accol{0, 1}$,
		\item jet relations generated by the following requirements:
		\begin{itemize}
			\item $(\loch, 0) : X \to X \multip \rival{\tej_i}$ is a jet set morphism,
			\item $(\loch, 1) : \Op_i(X) \to X \multip \rival{\tej_i}$ is a jet set morphism,
			\item $(x, 0) \tej_i (x, 1)$ for all $x \in X$.
		\end{itemize}
	\end{itemize}
	This defines the \textbf{opposite $i$-twisted prism functor} $\loch \multip \rival{\jet_i} : \jetset{\vec a} \to \jetset{\vec a}$.
	
	If $a_i = \npolar$, then we call this simply the \textbf{$i$-prism functor $\loch \multip \rival{\edge_i}$.}
\end{definition}
\todoi{Should there be a prism functor for $\rival{\equijet_i}$? It would only be well-defined on symmetric jet sets, the category of which is $\jetset{\vec a'}$ where $a'_i = \npolar$, so it is derived from $\loch \multip \rival{\edge_i}$ and the forgetful functor $U^\npolar_i$.}
\begin{corollary} \label{thm:twisted-prism-op}
	We have $\loch \multip \rival{\tej_i} = \Op_i(\loch \multip \rival{\jet_i})$. \qed
\end{corollary}

\subsection{Intermezzo: Cube Categories} \label{sec:cubes}
We introduce a family of cube categories with one flavour of dimension. Fix a monad $M$ on $\Set$.
\begin{example} \label{ex:cube-monads}
	Typically $M$ will be one of the following:
	\begin{itemize}
		\item The `exception' monad $\bipointed$ that sends a set $X$ to $X \uplus \accol{0,1}$, which is the carrier of the free bipointed set over $X$;
		\item The monad $\bipointedsym$ that sends a set $X$ to $X \uplus \set{\lnot x}{x \in X} \uplus \accol{0, 1}$, which is the carrier of the free [bipointed set equipped with an involution $\lnot$ that swaps $0$ and $1$] over $X$;
		\item The monad $\distlattice$ that sends a set $X$ to the carrier of the free distributive lattice over $X$;
		\item The monad $\demorgan$ that sends a set $X$ to the carrier of the free de Morgan algebra over $X$;
		\item The monad $\booleanalg$ that sends a set $X$ to the carrier of the free boolean algebra over $X$.
	\end{itemize}
\end{example}

\subsubsection{Cartesian Cubes}
\begin{definition}
	We construct the \textbf{(named) category of cartesian $M$-cubes} $\cubecat^\cartes_M$ (and $\namedcubecat^\cartes_M$ resp.) stepwise:
	\begin{itemize}
		\item The Kleisli category $\Kleisli(M)$ of $M$ has objects $\inKleisli{X}$ where $X$ is a set, and its morphisms $\inKleisli f : \inKleisli X \to \inKleisli Y$ are functions $f : X \to MY$.
		\item Of this, we take the opposite $\Kleisli(M)\op$. (This is the Lawvere theory corresponding to the monad $M$.)
		\item We define $\namedcubecat^\cartes_M$ as the full subcategory of $\Kleisli(M)\op$ on finite sets.
		\item We define $\cubecat^\cartes_M$ as a designate skeleton of $\namedcubecat_M$, e.g.\ the full subcategory of $\namedcubecat^\cartes_M$ on sets of the form $\accol{0, \ldots, n-1}$ with $n \geq 0$.
	\end{itemize}
\end{definition}
Objects of $\namedcubecat^\cartes_M$ will be denoted as tuples of names $(\dm{i}_0 : \IX, \ldots, \dm{i}_{n-1} : \IX)$ where $\IX$ is meaningless but conveys the intuition that we regard $\dm{i}_k$ as a value ranging over the interval (the cube given by the singleton object).
A morphism $\vfi : (\dm{i}_0 : \IX, \ldots, \dm{i}_{n-1} : \IX) \to (\dm{j}_0 : \IX, \ldots, \dm{j}_{m-1} : \IX)$ is then a function sending each $\dm{j}_k$ to an expression $\dm{j}_k \psub{\vfi} \in M\accol{\dm{i}_0, \ldots, \dm{i}_{n-1}}$.
The morphism $\vfi$ will also be denoted as $(\dm{j}_0 \psub \vfi / \dm{j}_0, \ldots, \dm{j}_{m-1} \psub \vfi / \dm{j}_{m-1})$.
The situation in $\cubecat^\cartes_M$ is the same except that we now regard the names $\dm{i}_k$ as De Bruijn indices.

\begin{corollary}
The categories $\cubecat^\cartes_M$ and $\namedcubecat^\cartes_M$ have finite products, given by finite coproducts of sets. \qed
\end{corollary}

\subsubsection{Affine Cubes}
If $T$ is a \emph{container} monad \cite{container-combinatorics}, i.e. a monad whose underlying functor is a container functor \cite{containers} of the form $TX = \Sigma(s : S).(P(s) \to X)$, then we define $T^\# X$ as the set of \emph{affine} expressions $\Sigma(s : S).(P(s) \hookrightarrow X)$, which is an endofunctor on the category $\Set^{\hookrightarrow}$ of sets and injective functions.
If $M$ is merely a \emph{quotient} of a container monad, i.e. $M$ is of the form $MX = TX/\sim_X$ with $T$ as above, then we define $M^\# X$ as the set of equivalence classes with an affine representant.

\begin{remark}
An important source of monads such as $M$ are monads specified by a syntactic algebraic theory \cite{manes-book,adamek-book,keml-diagrams}.
A syntactic algebraic theory specifies a set of operations $S_0$, assigns to each operation $s : S_0$ an arity $P_0(s)$ which is again a set, and subjects these to a set of axioms.%
\footnote{We use `syntactic algebraic theory' to refer to the syntactic presentation as described here, and `monad' and `Lawvere theory' to refer to the less syntactic objects they specify.
%There seems to be disagreement on the terminology.
%Syntactic algebraic theories (a.k.a. presentations of algebraic theories \cite{manes-book,keml-diagrams}) give rise to a monad and a Lawvere category (a.k.a. Lawvere theory or algebraic theory \cite{manes-book}).
%Depending on your definition of a Lawvere theory, the monadic presentation and the Lawvere theory may be interderivable; in turn, they give rise to a category of algebras.
%Neither of these two steps reflects equivalence, so we truly have three different notions, and there is disagreement which one the term `algebraic theory' refers to \cite{adamek-book}.
%
%Different syntactic algebraic theories may have the same category of models, in which case they are called \emph{Morita equivalent} \cite[ch.\ 15]{adamek-book}. We can thus distinguish algebraic theories by their syntactic presentation or consider them up to Morita equivalence. We will use \emph{syntactic algebraic theory} in the former case and \emph{monadic/Lawverian algebraic theory} in the latter case. Some authors insist on the usage of the word `presentation' for the former case \cite{manes-book}, or call the latter case `algebraic categories' \cite{adamek-book}, while the term `algebraic theories' seems to be in use for either case.
}
The container $(S_0, P_0)$ specifies a container functor $FX = \Sigma(s : S_0).(P_0(s) \to X)$ on $\Set$.
A free monad $F^*$ over this functor $F$ exists and satisfies the fixpoint equation $F^* X \cong X \uplus F F^* X$.
We remark that the free monad $F^*$ over a container functor $F$ is again a container functor, i.e.\ there exists a container $(S, P)$ such that
$F^* X = \Sigma(s : S).(P(s) \to X)$
specifies the free monad over $F$.
The axioms determine an equivalence relation $\sim_X$ on $F^* X$ such that $M X := F^* X/\sim_X$ is again a monad.
This situation applies to each of the monads in \cref{ex:cube-monads}.

In fact, often the quotient can be taken already at the level of the container, so that there exists a container $(S', P')$ such that $MX \cong \Sigma(s : S').(P'(s) \to X)$.
\end{remark}

We say that $(s, f), (s', f') \in T^\# X$ are \textbf{mutually fresh}, denoted $(s, f) \freshbin (s', f')$, if $f(P(s))$ and $f'(P(s'))$ are disjoint.
Elements of $M^\# X$ are mutually fresh if they have mutually fresh representants.
We call the monad $(T, \eta, \mu)$ \textbf{affine} if $\eta_X : X \to TX$ lands in $T^\# X$ for all $X$ and $\mu_X : TTX \to TX$ restricted to $(TT)^\# X$ (note that container functors are closed under composition) lands in $T^\# X$; and similar for $M$.

\begin{definition}
Let $M$ be a quotient of a container monad, and let it be affine.
We construct the \textbf{(named) category of affine $M$-cubes $\cubecat_M^\affine$} (and $\namedcubecat_M^\affine$ resp.) stepwise:
\begin{itemize}
	\item The affine Kleisli category $\Kleisli^\#(M)$ has objects $\inKleisli{X}$ where $X$ is a set, and its morphisms $\inKleisli f : \inKleisli X \to \inKleisli Y$ are functions $f : X \to M^\#Y$ such that for any $x \neq x'$ in $X$, we have $f(x) \freshbin f(x')$.
	Identity and composition are well-defined because $M$ is affine.
	\item Of this, we take the opposite $\Kleisli^\#(M)\op$.
	\item We define $\namedcubecat_M^\affine$ as the full subcategory of $\Kleisli^\#(M)\op$ on finite sets.
	\item We define $\cubecat_M^\affine$ as a designate skeleton of $\namedcubecat^\affine_M$, e.g.\ the full subcategory of $\namedcubecat^\affine_M$ on sets of the form $\accol{0, \ldots, n-1}$ with $n \geq 0$.
\end{itemize}
\end{definition}
Objects will be represented as for the cartesian cube categories.

\begin{corollary}
The categories $\cubecat^\affine_M$ and $\namedcubecat^\affine_M$ have a symmetric monoidal structure $(\top, {*})$ given by finite coproducts of sets. The binary operation is called the \textbf{separating product}. \qed
\end{corollary}

\subsubsection{Examples}
This way, we get -- among others -- the following cube categories:
\begin{enumerate}
	\item[$\cubecat^\cartes_{\bipointed}$] The cartesian cube category.
	A morphism $\vfi : V \to W$ sends every dimension $\dm{j} \in W$ to $\dm{j} \psub \vfi \in V \cup \accol{0, 1}$.
	Its cubes have diagonals.
	\item[$\cubecat^\affine_{\bipointed}$] The affine cube category.
	A morphism $\vfi : V \to W$ sends every dimension $\dm{j} \in W$ to $\dm{j} \psub \vfi \in V \cup \accol{0, 1}$, such that if $\dm{j}\psub \vfi = \dm{j}' \psub \vfi \in V$ then $\dm{j} = \dm{j}'$.
	Its cubes have no diagonals.
	\item[$\cubecat^\cartes_{\bipointedsym}$] The symmetric cartesian cube category. We have a negation/involution/symmetry $(\lnot \dm{i} / \dm{j}) : (\dm i : \IX) \to (\dm{j} : \IX)$.
	\item[$\cubecat^\cartes_{\distlattice}$] The cartesian cube category with connections. We have morphisms $(\dm i \vee \dm{j} / \dm k), (\dm i \wedge \dm{j} / \dm k) : (\dm i : \IX, \dm{j} : \IX) \to (\dm k : \IX)$. There are no symmetries
	\item[$\cubecat^\cartes_{\demorgan}$] The CCHM cube category, which combines symmetries and connections \cite{cubical}. We have $(\dm i \wedge \lnot \dm i / \dm j) \neq (0/\dm j) : (\dm i : \IX) \to (\dm j : \IX)$ and $(\dm i \vee \lnot \dm i / \dm j) \neq (1/\dm j) : (\dm i : \IX) \to (\dm j : \IX)$.
	\item[$\cubecat^\cartes_{\booleanalg}$] A cube category very similar to the CCHM one, but we have $(\dm i \wedge \lnot \dm i / \dm j) = (0/\dm j) : (\dm i : \IX) \to (\dm j : \IX)$ and $(\dm i \vee \lnot \dm i / \dm j) = (1/\dm j) : (\dm i : \IX) \to (\dm j : \IX)$.
\end{enumerate}
We remark that $\cubecat^\affine_{\demorgan}$ and $\cubecat^\affine_{\booleanalg}$ should be isomorphic as the additional law of boolean algebras w.r.t.\ de Morgan algebras only affects non-affine expressions.

\subsubsection{The Endpoint Model}
We remarked above that $\catL := \Kleisli(M)\op$ is the Lawvere category of $M$.
It is known then (see e.g.\ \cite{keml-diagrams}), that the Eilenberg-Moore category of $M$ (which is the category of Eilenberg-Moore algebras of $M$) is equivalent to the category of product-preserving functors $\catL \to \Set$.
Such functors are fully determined by the image of the singleton set (as every set is a coproduct of singletons and the Kleisli-category retains coproducts) and that image will be exactly the carrier of the corresponding Eilenberg-Moore algebra.

It is clear that both the cartesian and affine (named) cube categories are subcategories of $\catL$.
As such, any $M$-algebra induces a functor $\catL \to \Set$ and hence a functor from any of the $M$-cube categories to $\Set$.

We are interested, in particular, in $M$-algebra structures on the set of endpoints $\accol{0, 1}$.
Each of the monads in \cref{ex:cube-monads} has an obvious such algebra structure.
Thus, we get functors from each of the corresponding cube categories to $\Set$,
sending $(\dm{i}_0 : \IX, \ldots, \dm{i}_{n-1} : \IX)$ to $\accol{0, 1}^{\accol{\dm{i}_0, \ldots, \dm{i}_{n-1}}}$.
We will denote each of those functors as $\EP$.

Recall that a morphism $\vfi : (\dm{i}_0 : \IX, \ldots, \dm{i}_{n-1} : \IX) \to (\dm{j}_0 : \IX, \ldots, \dm{j}_{m-1} : \IX)$ assigns to each $\dm j$ a value $\dm j\psub \vfi$ in $M \accol{\dm{i}_0, \ldots, \dm{i}_{n-1}}$, the free $M$-algebra over $\accol{\dm{i}_0, \ldots, \dm{i}_{n-1}}$.
The function $\EP(\vfi)$ is defined by
\[
	\EP(\vfi)
	\paren{ v^{\accol{\dm{i}_0, \ldots, \dm{i}_{n-1}} \to \accol{0, 1}} }
	(\dm j)
	= \alpha\paren{ M(v)(\dm j \psub \vfi) },
\]
where $\alpha : M\accol{0, 1} \to \accol{0, 1}$ is the algebra structure on $\accol{0, 1}$.
Using the operation $\bindop^\alpha : MX \to (X \to \accol{0, 1}) \to \accol{0, 1} : \hat x \mapsto f \mapsto \alpha (Mf(\hat x))$, we can write this as $\EP(\vfi)(v)(\dm j) = \dm j \psub \vfi \bindop^\alpha v$.

\begin{proposition}
	The functor $\EP : \cubecat_\booleanalg \to \Set$ is fully faithful.
\end{proposition}
\begin{proof}
	We need to show that any function $f : \accol{0, 1}^{\accol{\dm{i}_0, \ldots, \dm{i}_{n-1}}} \to \accol{0, 1}^{\accol{\dm{j}_0, \ldots, \dm{j}_{m-1}}}$
	can be obtained as some $\EP(\vfi)$ with $\vfi : (\dm{i}_0 : \IX, \ldots, \dm{i}_{n-1} : \IX) \to (\dm{j}_0 : \IX, \ldots, \dm{j}_{m-1} : \IX)$.
	We remark that such a function $f$ in fact consists of $m$ truth tables in $n$ boolean variables.
	From the full disjunctive normal form, it is clear that elements of the free boolean algebra are in 1-1 correspondence with truth tables.
	Concretely, for each $\dm j$, define $\dm j \psub \vfi$ to be the element of $\booleanalg \accol{\dm{i}_0, \ldots, \dm{i}_{n-1}}$ corresponding to the truth table $f(\loch, \dm j)$.
	Then $\dm j \psub \vfi \bindop^\alpha v$ will evaluate $\dm j \psub \vfi$ after replacing each variable $\dm i$ with its value $v(\dm i)$, yielding the value $f(v, \dm j)$ prescribed by the truth table $f(\loch, \dm j)$.
\end{proof}

\subsection{Jet Cubes}

\subsubsection{Jet Cube Objects}
\begin{definition} \label{def:jet-cube}
	Let $\vec a$ be a mask.
	We define the set of \textbf{jet cubes of mask $\vec a$} essentially as the set of lists of degrees $0 \leq i < \lenpol{\vec a}$, but we denote them as
	$(\dm i_0 : \rival{\jet_{i_0}}, \ldots, \dm i_{n-1} : \rival{\jet_{i_{n-1}}})$, thinking of the names $\dm{i}_k$ as De Bruijn indices.
	If $a_i = \npolar$, we write $\rival{\edge_i}$ instead of $\rival{\jet_i}$.
	
	More generally, we define the set of \textbf{opposable jet cubes of mask $\vec a$} as the set of lists of elements of $\set{\jet_i, \tej_i}{a_i = \ypolar} \cup \set{\edge_i}{a_i = \npolar}$, but we denote them as
	$(\dm i_0 : \rival{\jet_{i_0}}, \ldots, \dm i_{n-1} : \rival{\tej_{i_{n-1}}})$.
	If $a_i = \npolar$, we use the symbols $\rival{\jet_i}$, $\rival{\tej_i}$ and $\rival{\edge_i}$ interchangeably.
%	the set of objects of $\jetset{\vec a}$ generated by:
%	\begin{itemize}
%		\item the terminal object $\top := \accol{()}$,
%		\item the twisted prism operations $\loch \multip \rival{\jet_i} : \Obj(\jetset{\vec a}) \to \Obj(\jetset{\vec a})$.
%	\end{itemize}
%	We will also denote such cubes as $(\dm i_0 : \rival{\jet_{i_0}}, \ldots, \dm i_{n-1} : \rival{\jet_{i_{n-1}}})$, thinking of the names $\dm{i}_k$ as De Bruijn indices.
\end{definition}
%It is clear that every jet cube can be generated by the above steps in only a single way, so we can reason about jet cubes inductively.%
%\footnote{In a proof assistant, we would define an inductive predicate over objects of $\jetset{\vec a}$ and prove that it is a mere proposition.}
\begin{definition} \label{def:jep}
	For any (opposable) jet cube $W$, we define the jet set $\JEP(W)$ by iteratively applying the ((opposite) twisted) prism functors:
	\begin{align*}
		\JEP(()) &= \top \\
		\JEP(W, \dm i : \rival{\jet_i}) &= \JEP(W) \multip \rival{\jet_i} \\
		\JEP(W, \dm i : \rival{\tej_i}) &= \JEP(W) \multip \rival{\tej_i} \\
		\JEP(W, \dm i : \rival{\edge_i}) &= \JEP(W) \multip \rival{\edge_i}
	\end{align*}
\end{definition}
\begin{definition} \label{def:jet-cube-erasure}
	We define the \textbf{jet-erasure function} $\erasejc{\loch}$, which sends (opposable) jet cubes of any mask to cubes (i.e. objects of any of the cube categories defined in \cref{sec:cubes}), by
	\[
		\erasejc{()} = (),
		\qquad \qquad
		\erasejc{(W, \dm i : \rival{\jet_i})} =
		\erasejc{(W, \dm i : \rival{\tej_i})} =
		\erasejc{(W, \dm i : \rival{\edge_i})} =
		(\erasejc{W}, i : \IX).
	\]
\end{definition}
\begin{corollary} \label{thm:jet-cube-pullback}
	For any (opposable) jet cube $W$, the carrier of $\JEP(W)$ is $\EP(\erasejc{W})$.
	Thus, every (opposable) jet cube determines an object of the following strict pullback of categories:
	\[
			\xymatrix{
				\cubecat^\eps_M \times_\Set \jetset{\vec a}
					\ar@{.>}[rr]^{\JEP}
					\ar@{.>}[d]_{\erasejc{\loch}}
					\xydrcorner
				&&
				\jetset{\vec a}
					\ar[d]^U
				\\
				\cubecat^\eps_M
					\ar[rr]_{\EP}
				&&
				\Set,
			}
	\]
	It is straightforward to see that this object is uniquely determined. \qed
\end{corollary}

\subsubsection{Jet Cube Categories}
\begin{definition} \label{def:jet-cube-cat}
	Let $\vec a$ be a mask and $\eps \in \accol{\affine, \cartes}$.
	We define the category $\jetcubeopt{\eps}{M}{\vec a}$ (or $\ojetcubeopt{\eps}{M}{\vec a}$) of \textbf{(opposable) affine/cartesian jet $M$-cubes of mask $\vec a$} as the full subcategory of $\cubecat^\eps_M \times_\Set \jetset{\vec a}$ on (opposable) jet cubes, as justified by \cref{thm:jet-cube-pullback}.
\end{definition}
%\begin{definition} \label{def:jet-cube-cat}
%	Let $\vec a$ be a mask and $\eps \in \accol{\affine, \cartes}$.
%	We define the category $\jetcubeopt{\eps}{M}{\vec a}$ of \textbf{affine/cartesian jet $M$-cubes of mask $\vec a$} as the category
%	\begin{itemize}
%		\item whose objects are jet cubes of mask $\vec a$,
%		\item whose Hom-sets are given by the pullback
%		\begin{equation}
%			\xymatrix{
%				\Hom_{\jetcubeopt{\eps}{M}{\vec a}}(V, W)
%					\ar@{.>}[rr]^{\JEP}
%					\ar@{.>}[d]_{\erasejc{\loch}}
%					\xydrcorner
%				&&
%				\Hom_{\jetset{\vec a}}(\JEP(V), \JEP(W))
%					\ar[d]^U
%				\\
%				\Hom_{\cubecat^\eps_M}(\erasejc V, \erasejc W)
%					\ar[rr]_{\EP}
%				&&
%				\Hom_{\Set}(\EP(\erasejc V), \EP(\erasejc W)),
%			}
%		\end{equation}
%		i.e.\ they are morphisms between cubes that respect the jet relations on endpoints.
%	\end{itemize}
%\end{definition}
\begin{definition} \label{def:prism-cube}
	Let $\vec a$ be a mask, $\eps \in \accol{\affine, \cartes}$ and $i < \lenpol{\vec a}$. We define the \textbf{$i$-twisted prism functor $\loch \multip (\dm i : \rival{\jet_i}) : \jetcubeopt{\eps}{M}{\vec a} \to \jetcubeopt{\eps}{M}{\vec a}$} on jet cubes as follows:\footnote{We still think of $\dm i$ as a De Bruijn index.}
	\begin{itemize}
		\item The object $W$ is sent to $(W, \dm i : \rival{\jet_i})$.
		\item The morphism $\vfi : V \to W$ is sent to the morphism $\vfi \multip (\dm i : \rival{\jet_i})$ such that
		\begin{itemize}
			\item If $\eps = \affine$ then $\erasejc{\vfi \multip (\dm i : \rival{\jet_i})} = \erasejc \vfi * (\dm i : \IX)$,
			\item If $\eps = \cartes$ then $\erasejc{\vfi \multip (\dm i : \rival{\jet_i})} = \erasejc \vfi \times (\dm i : \IX)$,
			\item $\JEP(\vfi \multip (\dm i : \rival{\jet_i})) = \JEP(\vfi) \multip \rival{\jet_i}$.
		\end{itemize}
	\end{itemize}
	If $a_i = \npolar$, then we call this simply the \textbf{$i$-prism functor} $\loch \multip (\dm i : \rival{\edge_i}) : \jetcubeopt{\eps}{M}{\vec a} \to \jetcubeopt{\eps}{M}{\vec a}$.
	
	We similarly define, on opposable jet cubes:
	\begin{itemize}
		\item the $i$-twisted prism functor $\loch \multip (\dm i : \rival{\jet_i}) : \jetcubeopt{\eps}{M}{\vec a} \to \jetcubeopt{\eps}{M}{\vec a}$ when $a_i = \ypolar$,
		\item the opposite $i$-twisted prism functor $\loch \multip (\dm i : \rival{\tej_i}) : \jetcubeopt{\eps}{M}{\vec a} \to \jetcubeopt{\eps}{M}{\vec a}$ when $a_i = \ypolar$,
		\item the $i$-prism functor $\loch \multip (\dm i : \rival{\edge_i}) : \jetcubeopt{\eps}{M}{\vec a} \to \jetcubeopt{\eps}{M}{\vec a}$ when $a_i = \npolar$.
	\end{itemize}
\end{definition}
\begin{proposition} \label{thm:prism-commute}
	((Opposable) twisted) prism functors for different degrees commute.
\end{proposition}
\begin{proof}
	On jet-erased cubes, we have an isomorphism
	\begin{description}
		\item[\framebox{$\eps = \affine$}] $(\dm i/\dm i, \dm j/\dm j) \quad:\quad \loch * (\dm i : \IX) * (\dm j : \IX) \quad\cong\quad \loch * (\dm j : \IX) * (\dm i : \IX)$,
		\item[\framebox{$\eps = \cartes$}] $(\dm i/\dm i, \dm j/\dm j) \quad:\quad \loch \times (\dm i : \IX) \times (\dm j : \IX) \quad\cong\quad \loch \times (\dm j : \IX) \times (\dm i : \IX)$.
	\end{description}
	One easily checks that this isomorphism respects the jet relations.
\end{proof}
\begin{proposition}
	The involution $\Op_i : \jetset{\vec a} \to \jetset{\vec a}$ lifts to an involution $\Opcube_i : \ojetcubeopt{\eps}{M}{\vec a} \to \ojetcubeopt{\eps}{M}{\vec a}$ which turns around the last arrow of degree $i$.
\end{proposition}
\begin{proof}
	The functor $\Op_i$ is clearly defined on the pullback $\cubecat^\eps_M \times_\Set \jetset{\vec a}$, so we need to prove that it sends opposable jet cubes to opposable jet cubes.

	If $a_i = \npolar$, then $\Op_i = \Id$ trivially lifts as the identity functor.
	Since in that case $\rival{\jet_i}$ and $\rival{\tej_i}$ both mean $\rival{\edge_i}$, turning around the last arrow of degree $i$ is also the identity functor.
	
	If $a_i = \ypolar$, it is easy to see that $\Op_i$ commutes with twisted prism functors (on jet sets) for different degrees. When it meets a twisted prism functor of the same degree, it reverses the direction of the arrow (\cref{thm:twisted-prism-op}):
	\[
		\Op_i(X \multip \rival{\jet_i}) = X \multip \rival{\tej_i},
		\qquad
		\Op_i(X \multip \rival{\tej_i}) = X \multip \rival{\jet_i}.
	\]
	It is then easy to see that $\Opcube_i$ acts on opposable jet cubes by reversing the last arrow of degree $i$.
\end{proof}

\subsubsection{A Calculus of Opposable Jet Cube Morphisms}
In this section, we develop a calculus that inductively generates the morphisms of the category $\ojetcubeopt{\eps}{M}{\vec a}$ and therefore also those of its full subcategory $\jetcubeopt{\eps}{M}{\vec a}$.

\begin{convention}
	When presenting the calculus, with permission from \cref{thm:prism-commute}, we will order the dimensions of an opposable jet cube by \emph{decreasing} degree.
\end{convention}
\begin{remark}
	We will only consider the monads $\bipointedsym$ and $\booleanalg$ because:
	\begin{itemize}
		\item We need involutions in order to be able to work with the source-side of the twisted prism,
		\item We do not see any advantage of $\demorgan$ over $\booleanalg$. In particular, we want $\EP$ to be faithful.
	\end{itemize}
\end{remark}
\todoi{Define $\shp_i$ and $\shp_i^\ypolar$. But $\shp_i$ is not even a functor for $i \neq 0$. So we will include an equijet interval.}

Zero:
\[
	\inference{
		\vfi : V \to \Op_i(W)
	}{
		(\vfi, 0/\dm i) : V \to (W, i : \rival{\jet_i})
	}{}
	\qquad
	\inference{
		\vfi : V \to W
	}{
		(\vfi, 0/\dm i) : V \to (W, i : \rival{\tej_i})
	}{}
\]

One:
\[
	\inference{
		\vfi : V \to W
	}{
		(\vfi, 1/\dm i) : V \to (W, i : \rival{\jet_i})
	}{}
	\qquad
	\inference{
		\vfi : V \to \Op_i(W)
	}{
		(\vfi, 1/\dm i) : V \to (W, i : \rival{\tej_i})
	}{}
\],

Negation/involution/symmetry:
\[
	\inference{
		(\vfi, t) : V \to (W, i : \rival{\jet_i})
	}{
		(\vfi, \lnot t/\dm i) : V \to (W, i : \rival{\tej_i})
	}{}
	\qquad
	\inference{
		(\vfi, t) : V \to (W, i : \rival{\tej_i})
	}{
		(\vfi, \lnot t/\dm i) : V \to (W, i : \rival{\jet_i})
	}{}
\]

Preservation:
\[
	\inference{
		\vfi : V \to W
	}{
		(\vfi, \dm i / \dm i) : (V, \dm i : \rival{\jet_i}) \to (W, \dm i : \rival{\jet_i})
	}{}
	\qquad
	\inference{
		\vfi : V \to W
	}{
		(\vfi, \dm i / \dm i) : (V, \dm i : \rival{\tej_i}) \to (W, \dm i : \rival{\tej_i})
	}{}
\]

Companion:
\[
	\inference{
		\vfi : \shp^\ypolar_i(\shp^\ypolar_j U, V) \to W
	}{
		(\vfi, \dm j / \dm i) : (U, \dm j : \rival{\jet_j}, V) \to (W, \dm i : \rival{\jet_i})
	}{$j > i$}
\]
\[
	\inference{
		\vfi : \shp^\ypolar_i(\shp^\ypolar_j U, V) \to W
	}{
		(\vfi, \dm j / \dm i) : (U, \dm j : \rival{\tej_j}, V) \to (W, \dm i : \rival{\tej_i})
	}{$j > i$}
\]

Conjoint:
\[
	\inference{
		\vfi : \shp^\ypolar_i(\shp^\ypolar_j U, V) \to W
	}{
		(\vfi, \dm j / \dm i) : (U, \dm j : \rival{\tej_j}, V) \to (W, \dm i : \rival{\jet_i})
	}{$j > i$}
\]
\[
	\inference{
		\vfi : \shp^\ypolar_i(\shp^\ypolar_j U, V) \to W
	}{
		(\vfi, \dm j / \dm i) : (U, \dm j : \rival{\jet_j}, V) \to (W, \dm i : \rival{\tej_i})
	}{$j > i$}
\]

Forward conjunction and backward disjunction:
\[
	\inference{
		(\vfi, s/\dm i, t/\dm j) : V \to (W, \dm i : \rival{\jet_i}, \dm j : \rival{\jet_i})
		\qquad
		\dom(\vfi) = \shp^\ypolar_i V
	}{
		(\vfi, s \wedge t/\dm k) : V \to (W, \dm k : \rival{\jet_i})
	}{$\booleanalg$}
\]
\[
	\inference{
		(\vfi, s/\dm i, t/\dm j) : V \to (W, \dm i : \rival{\tej_i}, \dm j : \rival{\tej_i})
		\qquad
		\dom(\vfi) = \shp^\ypolar_i V
	}{
		(\vfi, s \vee t/\dm k) : V \to (W, \dm k : \rival{\tej_i})
	}{$\booleanalg$}
\]

Forward disjunction and backward conjunction:
\[
	\inference{
		(\vfi, s/\dm i, t/\dm j) : V \to (W, \dm i : \rival{\tej_i}, \dm j : \rival{\jet_i})
		\qquad
		\dom(\vfi) = \shp^\ypolar_i V
	}{
		(\vfi, s \vee t/\dm k) : V \to (W, \dm k : \rival{\jet_i})
	}{$\booleanalg$}
\]
\[
	\inference{
		(\vfi, s/\dm i, t/\dm j) : V \to (W, \dm i : \rival{\jet_i}, \dm j : \rival{\tej_i})
		\qquad
		\dom(\vfi) = \shp^\ypolar_i V
	}{
		(\vfi, s \wedge t/\dm k) : V \to (W, \dm k : \rival{\tej_i})
	}{$\booleanalg$}
\]
Note: the start of the Hamiltonian path of a twisted cube always has coordinates $(1, \ldots, 1, 0)$.

Truth table (cartesian):
\[
	\inference{
		a_i = \npolar
		\qquad
		t[x_1, \ldots, x_n] \in \booleanalg \accol{x_1, \ldots, x_n}
		\\
		(\vfi, s_1/\dm i_1, \ldots, s_n/\dm i_n) : V \to (W, \dm i_1 : \rival{\edge_i}, \ldots, \dm i_n : \rival{\edge_i})
	}{
		(\vfi, t[s_1, \ldots, s_n] / \dm i) : V \to (W, \dm i : \rival{\edge_i})
	}{$\booleanalg$, $\cartes$}
\]

\todoi{I think we can do any left associative series of conjunctions/disjunctions.}

\todoi{Connections:
\begin{itemize}
	\item Affine: traverse the boolean algebra syntax tree and analyze where the last variable is.
	\item Cartesian: regard it as an arbitrary truth table and spell it as $\dm i \mathbin ? \dm j \mathbin : \dm k$. This should be an increasing function of $\dm i$, i.e.\ $\dm k \Rightarrow \dm j$. Moreover, $\dm j$ should be covariant and $\dm k$ contravariant. (Your twisted prisms still don't have diagonals, but that's ok since the entire expression concerns one codomain dimension.)
\end{itemize}
}

\todoi{Conjecture: $\jetcubeaff{\bipointedsym}{[\ypolar]}$ is the twisted cube category \cite{pinyo-twisted}. Symmetries zijn nodig om aan 0-kant te embedden. Moogt ook cartesians pakken, want hebt toch geen diagonalen.}
\todoi{Use: $\jetcubeaff{\booleanalg}{\vec a}$ of $\jetcubecart{\booleanalg}{\vec a}$}
\todoi{Twisted prism functors at different degrees commute.}

\subsection{Jet Cubes Old}
\todoi{This is all wrong: there is no symmetry, no degradation and no connections/companions/conjoints.}

\begin{corollary} \label{thm:jet-cube-op}
	If $W$ is a jet cube, then $\Op_i(W)$ is isomorphic (in $\jetset{\vec a}$) to a jet cube $\Opcube_i(W)$.
	We have $\Opcube_i \circ \Opcube_i = \Id$.
	\todoi{Do we use this? I guess later on to characterize.}
\end{corollary}
In fact, it happens to be the case that $\Opcube_i(W) = W$, but we will avoid to use this fact as the isomorphism $\Op_i(W) \cong \Opcube_i(W) = W$ is not the identity on objects.
\begin{proof}
	We define recursively:
	\begin{itemize}
		\item $\Opcube_i(\top) := \top = \Op_i(\top)$,
		\item $\Opcube_i(W \multip \rival{\jet_i}) := W \multip \rival{\jet_i} \cong \Op_i(W \multip \rival{\jet_i})$ (by swapping 0 and 1),
		\item $\Opcube_i(W \multip \rival{\jet_j}) := \Opcube_i(W) \multip \rival{\jet_j} \cong \Op_i(W) \multip \rival{\jet_j} = \Op_i(W \multip \rival{\jet_j})$ if $i \neq j$. \qedhere
	\end{itemize}
\end{proof}

\begin{definition} \label{def:jet-cube-cat}
	Let $\vec a$ be a mask.
	We define the category $\jetcube{\vec a}$ of \textbf{jet cubes of mask $\vec a$} as the proof-relevant sub\emph{graph} (a priori; it will be a category after \cref{thm:jet-cube-cat}) of $\jetset{\vec a}$ whose nodes are the jet cubes, and whose edges are generated by:
	\begin{itemize}
		\item $(\vfi, 0) : \Hom_{\jetcube{\vec a}}(V, W \multip \rival{\jet_i})$ for $\vfi : \Hom_{\jetcube{\vec a}}(V, \Opcube_i(W))$,
		\item $(\vfi, 1) : \Hom_{\jetcube{\vec a}}(V, W \multip \rival{\jet_i})$ for $\vfi : \Hom_{\jetcube{\vec a}}(V, W)$,
		\item $\vfi \multip \rival{\jet_i} : \Hom_{\jetcube{\vec a}}(V \multip \rival{\jet_i}, W \multip \rival{\jet_i})$ for $\vfi : \Hom_{\jetcube{\vec a}}(V, W)$,
		\item $() : \Hom_{\jetcube{\vec a}}(V, \top)$.
	\end{itemize}
\end{definition}
Again, every jet cube morphism can be generated by the above steps in only a single way, so we can reason about them inductively.
\begin{corollary} \label{thm:jet-cube-op-ftr}
	Each $\Opcube_i$ extends to a well-defined graph endomorphism (and hence, after \cref{thm:jet-cube-cat}, an endofunctor) on $\jetcube{\vec a}$.
	We have $\Opcube_i \circ \Opcube_i = \Id$.
\end{corollary}
\begin{proof}
	We define recursively:
	\begin{itemize}
		\item $\Opcube_i(\vfi, 0 \in \rival{\jet_i}) := (\Opcube_i(\vfi), 1)$,
		\item $\Opcube_i(\vfi, 0 \in \rival{\jet_j}) := (\Opcube_i(\vfi), 0)$ if $i \neq j$,
		\item $\Opcube_i(\vfi, 1 \in \rival{\jet_i}) := (\Opcube_i(\vfi), 0)$,
		\item $\Opcube_i(\vfi, 1 \in \rival{\jet_j}) := (\Opcube_i(\vfi), 1)$ if $i \neq j$,
		\item $\Opcube_i(\vfi \multip \rival{\jet_i}) = \vfi \multip \rival{\jet_i}$,
		\item $\Opcube_i(\vfi \multip \rival{\jet_j}) = \Opcube_i(\vfi) \multip \rival{\jet_j}$ if $i \neq j$,
		\item $\Opcube_i(()) = ()$. \qedhere
	\end{itemize}
\end{proof}
\begin{corollary} \label{thm:jet-cube-cat}
	$\jetcube{\vec a}$ is a category.
\end{corollary}
\begin{proof}
	Identity and composition can be defined recursively; category laws proven inductively.
\end{proof}

\subsection{Comparison to Pinyo and Kraus's Twisted Cube Category}
\todoi{Compare to all the different cube categories they have, I think I have them all included.}
In this section, we discuss how our approach reduces to Pinyo and Kraus's \cite{pinyo-twisted} when $\vec a = [\ypolar]$.
$\jetset{[\ypolar]}$ is the category of proof-irrelevant reflexive graphs.
Pinyo and Kraus use arbitrary proof-irrelevant graphs, but since $\top$ is reflexive and the twisted prism functor \cite[def.\ 4]{pinyo-twisted} restricts to reflexive graphs, all twisted cubes are reflexive graphs anyway.
\todoi{Continue}


Out of the category of jet sets, we will carve the category of jet cubes, in such a way that the category of jet cubes of mask $[\ypolar]$ is isomorphic to the twisted cube category \cite{pinyo-twisted}.


\bibliographystyle{alphaurl}
\bibliography{../refs/refs.bib}

\end{document}