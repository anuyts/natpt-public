\documentclass[a4paper]{article}

\usepackage{amsthm}
\pdfoutput=1
\usepackage{libertine}
\usepackage[libertine]{newtxmath}
\usepackage{a4wide}

%\usepackage{mathpartir}
%\usepackage{natbib}
%\usepackage{tipa}

%\usepackage{geometry}
%\newcommand{\todo}[1]{\textbf{\red{*}}\marginpar{\red{#1}}}
%\renewcommand{\todoi}[1]{\par \textbf{\red{#1}} \par}

%\usepackage{graphicx}
%\usepackage{mathtools}
%\usepackage{amsmath}
%\usepackage{amssymb}
%\usepackage{thmtools}
%\usepackage{mathrsfs}
%%\usepackage{mathabx}
%\usepackage{latexsym}
\usepackage[hidelinks,bookmarksnumbered]{hyperref}
\usepackage{cleveref}
\usepackage{xcolor}
\usepackage{xypic}
	\CompileMatrices
\usepackage{stmaryrd}
%\usepackage{epstopdf}
%\usepackage{array}
%\usepackage[normalem]{ulem}

\usepackage[english]{babel}
%\usepackage[dutch]{babel}

\usepackage{../natpt-macros}

%\newcommand{\todoi}[1]{\textbf{\textcolor{red}{#1 \qed}}\par\noindent}
%\newcommand{\todo}[1]{\textbf{\textcolor{red}{\footnote{\textcolor{red}{#1}}}}}
%\newcommand{\todoi}[1]{}
%\newcommand{\todo}[1]{}

\newcommand{\thetitle}{Naturality Pretype Theory - Extended Version}
\newcommand{\theauthors}{Andreas Nuyts}
\begin{document}
	\addtolength{\voffset}{-.5in}

\title{\thetitle}
%\subtitle{Technical report}
\date{\today}
\author{\theauthors{}}
%\address{imec-DistriNet, KU Leuven, Belgium}
\maketitle
%\vspace*{-.5in}
%\begin{abstract}
%\end{abstract}

\tableofcontents

\pagebreak

\section{Base Categories and Modes}

\subsection{Modes are Anpolarity Lists}
In RelDTT, modes were natural numbers (minus one) expressing the number of available relations.
In NatPT, we will specify for each of these relations whether it is directed or not.
\begin{definition} \label{def:anpolarity}
	An \textbf{anpolarity}\footnote{`An' is Latin for `whether', as in `Nescio an polare sit,' meaning `I do not know whether it is polar'.} is an element of the set $\Anpolarity := \accol{\ypolar, \npolar}$.
\end{definition}
\noindent As our set of modes, we take $\List\,\Anpolarity$. If $\vec a \in \List\,\Anpolarity$, we write $\lenpol{\vec a}$ for its length.

\begin{remark}
	We note that it is always possible to restrict our mode theory, by discarding modes but keeping the same modalities and 2-cells between remaining modes.
	We could decide to restrict to any of the following subsets of modes:
	\begin{itemize}
		\item Modes of the form $[\ypolar]^*$, i.e.\ where all degrees are polar,
		\item Modes of the form $([\npolar] | [\npolar, \ypolar])^*$, i.e.\ where we think of a level as containing a path relation and optionally a weaker jet relation,
		\begin{itemize}
			\item Modes of the form $[\npolar, \ypolar]^*$ where the presence of a jet relation at each level is required,
		\end{itemize}
		\item Modes of the form $([\npolar] | [\ypolar, \npolar])^*$, i.e.\ where we think of a level as containing a bridge relation and optionally a stronger jet relation,
		\begin{itemize}
			\item Modes of the form $[\ypolar, \npolar]^*$ where the presence of a jet relation at each level is required,
		\end{itemize}
		\item Modes of the form $([\npolar, \npolar] | [\npolar, \ypolar, \npolar])^*$, i.e.\ where we think of a level as containing a path relation and a weaker bridge relation and optionally, in between, a jet relation,
		\begin{itemize}
			\item Modes of the form $[\npolar, \ypolar, \npolar]^*$ where the presence of a jet relation at each level is required.
		\end{itemize}
	\end{itemize}
	We will occasionally discuss these subtheories. By considering all of $\List\,\Anpolarity$ in the current paper, we maintain generality.
\end{remark}

\subsection{Jet Sets}
\begin{definition} \label{def:jetset}
	Let $\vec a$ be a mode. A \textbf{jet set of mode $\vec a$} is a set $X$ equipped with $\lenpol{\vec a}$ (proof-irrelevant) pre-orders $\jet_i$ where
	\begin{itemize}
		\item $0 \leq i < \lenpol{\vec a}$ is called the \textbf{degree} and
		\item $\jet_i$ is called the \textbf{$i$-jet relation}
	\end{itemize}
	such that
	\begin{itemize}
		\item when $a_i = \npolar$, then $\jet_i$ is an equivalence relation (i.e.\ symmetric), in which case we will denote it as $\bridge_i$ and call it the \textbf{$i$-edge relation} (notwithstanding that we still consider it a special case of a jet relation),
		\item $x \jet_i y$ implies both $x \jet_{i+1} y$ and $x \tej_{i+1} y$ whenever $0 \leq i < i+1 < \lenpol{\vec a}$.
	\end{itemize}
	A \textbf{morphism} of jet sets of mode $\vec a$ is a function that preserves all the jet and edge relations.
\end{definition}
\begin{proposition}
	Let $X$ be a jet set of mode $\vec a$ and $0 \leq i < j < \lenpol{\vec a}$.
	Then the double category whose objects are elements of $X$, morphisms are (unique) proofs of $x \jet_i y$, pro-arrows are (unique) proofs of $x \jet_{j} y$ and squares are elements of the unit type, is a pro-arrow equipment.
\end{proposition}
\begin{proof}
	The existence of companions and conjoints is trivial.
\end{proof}
\begin{definition}
	We define the
	\begin{itemize}
		\item \textbf{$i$-equijet relation} $\equijet_i$ as the symmetric interior of $\jet_i$, i.e.\ $x \equijet_i y$ if and only if $x \jet_i y$ and $x \tej_i y$;
		\item \textbf{$i$-ripple relation} $\ripple_i$ as the symmetric transitive closure of $\jet_i$.
	\end{itemize}
\end{definition}
It is immediately clear that for nonpolar degrees, the jet/edge, equijet and ripple relations coincide.
In general, we can observe that $x \ripple_i y$ implies $x \equijet_j y$ for $i < j$.
So for mode $[\ypolar, \ypolar, \npolar]$, we get
\[
\xymatrix{
	& 
	\jet_0
		\ar@{:>}[rd]
	&&& 
	\jet_1 
		\ar@{:>}[rd]
	&&& 
	\jet_2 
		\ar@{<:>}[rd]
	\\
	\equijet_0
		\ar@{:>}[ru]
		\ar@{:>}[rd]
	&&
	\ripple_0
		\ar@{:>}[r]
	&
	\equijet_1
		\ar@{:>}[ru]
		\ar@{:>}[rd]
	&&
	\ripple_1
		\ar@{:>}[r]
	&
	\equijet_2
		\ar@{<:>}[ru]
		\ar@{<:>}[rd]
	&
	\bridge_2
		\ar@{<:>}[l]
		\ar@{<:>}[u]
		\ar@{<:>}[r]
		\ar@{<:>}[d]
	&
	\ripple_2.
	\\
	& 
	\tej_0 
		\ar@{:>}[ru]
	&&& 
	\tej_1 
		\ar@{:>}[ru]
	&&& 
	\tej_2 
		\ar@{<:>}[ru]
}
\]

\subsection{Jet Cubes}
Out of the category of jet sets, we will carve the category of jet cubes, in such a way that the category of jet cubes of mode $[\ypolar]$ is isomorphic to the twisted cube category \cite{nlab:proarrow,proarrow1,proarrow2}.


\bibliographystyle{alphaurl}
\bibliography{../refs/refs.bib}

\end{document}